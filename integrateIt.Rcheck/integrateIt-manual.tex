\nonstopmode{}
\documentclass[a4paper]{book}
\usepackage[times,inconsolata,hyper]{Rd}
\usepackage{makeidx}
\usepackage[utf8]{inputenc} % @SET ENCODING@
% \usepackage{graphicx} % @USE GRAPHICX@
\makeindex{}
\begin{document}
\chapter*{}
\begin{center}
{\textbf{\huge Package `integrateIt'}}
\par\bigskip{\large \today}
\end{center}
\begin{description}
\raggedright{}
\inputencoding{utf8}
\item[Type]\AsIs{Package}
\item[Title]\AsIs{Integrates Functions Using Simpson or Trapezoid Methods}
\item[Version]\AsIs{0.1.0}
\item[Author]\AsIs{Jonah Klein-Barton}
\item[Maintainer]\AsIs{}\email{jonahkleinbarton@gmail.com}\AsIs{}
\item[Description]\AsIs{Uses the Simpson and/or Trapezoid methods to approximate the integrals of bounded functions}
\item[License]\AsIs{CC0}
\item[Collate]\AsIs{'Trapezoid.R' 'Simpson.R' 'integrateIt.R' 'print.R'}
\item[Encoding]\AsIs{UTF-8}
\item[LazyData]\AsIs{true}
\item[Suggests]\AsIs{testthat}
\item[RoxygenNote]\AsIs{6.0.1}
\end{description}
\Rdcontents{\R{} topics documented:}
\inputencoding{utf8}
\HeaderA{integrateIt}{integrateIt}{integrateIt}
%
\begin{Description}\relax
integrateIt
\end{Description}
%
\begin{Usage}
\begin{verbatim}
integrateIt(X, Y, bounds, Rule)
\end{verbatim}
\end{Usage}
%
\begin{Arguments}
\begin{ldescription}
\item[\code{X}] A list of even partitions of the bounds that the integral is beign taken on.

\item[\code{Y}] A numeric list with the same dimensionality as \code{x}, equal to the height of the function at each X value. X and Y form sets of ordered pairs.

\item[\code{bounds}] The minimum and maximum X values.

\item[\code{Rule}] Can be either "Trap" or "Simpson". Determines which form of estimation the method will use to estimate the integral
\end{ldescription}
\end{Arguments}
%
\begin{Value}
An object of class Squares containing
\begin{ldescription}
\item[\code{object}] An object of class Simpson or Trapezoid
\end{ldescription}
\end{Value}
%
\begin{Note}\relax
This is a very simple function
\end{Note}
%
\begin{Author}\relax
Jonah Klein-Barton
\end{Author}
%
\begin{Examples}
\begin{ExampleCode}

myX <- c(1,2,3,4,5)
myY <- c(1,4,9,16,25)
integrateIt(myX, myY, c(1,5), "Trap")
\end{ExampleCode}
\end{Examples}
\inputencoding{utf8}
\HeaderA{print,Trapezoid-method}{print}{print,Trapezoid.Rdash.method}
%
\begin{Description}\relax
print
\end{Description}
%
\begin{Usage}
\begin{verbatim}
## S4 method for signature 'Trapezoid'
print(x)
\end{verbatim}
\end{Usage}
%
\begin{Arguments}
\begin{ldescription}
\item[\code{x}] An object of class Trapezoid or Simpson
\end{ldescription}
\end{Arguments}
%
\begin{Value}
An estimation of the integral
\end{Value}
%
\begin{Note}\relax
This is a very simple function
\end{Note}
%
\begin{Author}\relax
Jonah Klein-Barton
\end{Author}
\inputencoding{utf8}
\HeaderA{Simpson-class}{An estimated integral of a function}{Simpson.Rdash.class}
%
\begin{Description}\relax
Object of class \code{SquaresPack} are created by the \code{addSquares} and \code{subtractSquares} functions
\end{Description}
%
\begin{Details}\relax
An object of the class `Simpson' has the following slots:
\begin{itemize}

\item \code{bounds} The lower bound and upper bound of the intergrand
\item \code{X} An ordered list of X values, between a and b
\item \code{Y} An ordered list of Y values, where Yn = F(Xn)
\item \code{integral} An estimate of the integral

\end{itemize}

\end{Details}
%
\begin{Author}\relax
Jonah Klein-Barton: \email{jonahkleinbarton@gmail.com}
\end{Author}
\inputencoding{utf8}
\HeaderA{Trapezoid-class}{An estimated integral of a function}{Trapezoid.Rdash.class}
%
\begin{Description}\relax
Object of class \code{SquaresPack} are created by the \code{addSquares} and \code{subtractSquares} functions
\end{Description}
%
\begin{Details}\relax
An object of the class `Trapezoid' has the following slots:
\begin{itemize}

\item \code{bounds} The lower bound and upper bound of the intergrand
\item \code{X} An ordered list of X values, between a and b
\item \code{Y} An ordered list of Y values, where Yn = F(Xn)
\item \code{integral} An estimate of the integral

\end{itemize}

\end{Details}
%
\begin{Author}\relax
Jonah Klein-Barton: \email{jonahkleinbarton@gmail.com}
\end{Author}
\printindex{}
\end{document}
